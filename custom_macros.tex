\usepackage{framed}
\usepackage{ifthen}
\usepackage[x11names]{xcolor}
\usepackage{soul}
%\definecolor{ForestGreen}{HTML}{228B22}

%%% Referencing Commands

\newcommand{\Chapref}[1]{{Chapter~\ref{#1}}}                                                   
\soulregister\Chapref{7}
\newcommand{\chapref}[1]{{Chap.~\ref{#1}}}                                                   
\soulregister\chapref{7}
\newcommand{\Figref}[1]{{Figure~\ref{#1}}}                                                   
\soulregister\Figref{7}
\newcommand{\figref}[1]{{Fig.~\ref{#1}}}                                                   
\soulregister\figref{7}
\newcommand{\secref}[1]{{Section~\ref{#1}}}
\soulregister\secref{7}
\newcommand{\nbcite}[1]{~\cite{#1}}
\soulregister\nbcite{7}

%%% Math spacing
% For comparison, the existing overlap macros:
% \def\llap#1{\hbox to 0pt{\hss#1}}
% \def\rlap#1{\hbox to 0pt{#1\hss}}
\def\clap#1{\hbox to 0pt{\hss#1\hss}}
\def\mathllap{\mathpalette\mathllapinternal}
\def\mathrlap{\mathpalette\mathrlapinternal}
\def\mathclap{\mathpalette\mathclapinternal}
\def\mathllapinternal#1#2{\llap{$\mathsurround=0pt#1{#2}$}}
\def\mathrlapinternal#1#2{\rlap{$\mathsurround=0pt#1{#2}$}}
\def\mathclapinternal#1#2{\clap{$\mathsurround=0pt#1{#2}$}}

%%% Placeholder macros

%\let\oldcite\cite
%\renewcommand{\cite}[1]{\ifthenelse{\equal{#1}{}}{\textsuperscript{\color{Firebrick3}[citation-needed]}}{\oldcite{#1}}}

\newcommand{\VAL}[1]{{\color{blue}\framebox[1.1\width]{\textbf{VALUE}}~}}
\def\REF{{\color{Firebrick3}\framebox[1.1\width]{REF}~}}
\def\CITE{{\color{Firebrick3}\framebox[1.1\width]{CITE}~}}

\newcommand{\TODO}[1]{{\color{Firebrick3}\large{\textbf{TODO: #1}}\normalsize}}
\newcommand{\PLACEHOLDER}[1]{{\textit{\color{Firebrick3}#1}}}
\newcommand{\BOXCOMMENT}[1]{{\color{Green4}\begin{framed}\noindent\emph{#1}\end{framed}}}
\newcommand{\COMMENT}[1]{{\color{Green4}\emph{#1}}}
\newcommand{\NOTE}[1]{\BOXCOMMENT{N.B. \\NOTE deprecated use \\BOXCOMMENT#1}}
%%% Math Commands

%General math commands

\newcommand{\unit}[1]{\ensuremath{\ \mathrm{#1}}}
\renewcommand{\vec}[1]{\mathbf{#1}}
\newcommand{\svec}[1]{\ensuremath{\bar{#1}}}
\newcommand{\uvec}[1]{\hat{#1}}
\newcommand*{\expE}{\unit{e}}

\newcommand{\pd}[1]{\ensuremath{\operatorname{\partial}\!{#1}}} 
\newcommand{\npd}[2][2]{\ensuremath{\operatorname{\partial}^{#1}\!{#2}}} 
\newcommand{\dd}[1]{\ensuremath{\operatorname{d}\!{#1}}} 
\renewcommand{\d}[1]{\ensuremath{\operatorname{d}\!{#1}}} 
\DeclareRobustCommand{\nd}[2][2]{\ensuremath{\operatorname{d}^{#1}\!{#2}}} 

%bra - ket
\newcommand{\bra}[1]{\ensuremath{\left<#1\right|}}
\newcommand{\ket}[1]{\ensuremath{\left|#1\right>}}

% angle bracket average
\newcommand{\avg}[1]{\ensuremath{\left<#1\right>}}

% complex numbers
\renewcommand{\imath}{\mathit{i}}
\renewcommand{\Re}{\mathbf{Re}}
\renewcommand{\Im}{\mathbf{Im}}

% derivatives
\newcommand{\pdiv}[2]{\ensuremath{\frac{\pd{#1}}{\pd{#2}}}}
\newcommand{\npdiv}[3][2]{\ensuremath{\frac{\npd[#1]{#2}}{\npd{#3^#1}}}}
\newcommand{\tdiv}[2]{\ensuremath{\frac{\d{#1}}{\d{#2}}}}
\newcommand{\ntdiv}[3][2]{\ensuremath{\frac{\protect\nd[#1]{#2}}{\d{#3^#1}}}}
%vector operators
\newcommand{\vgrad}[1]{\ensuremath{\nabla} #1}
\newcommand{\vdiv}[1]{\ensuremath{\nabla \cdot #1}}
\newcommand{\vcurl}[1]{\ensuremath{\nabla \times #1}}
\newcommand{\vlap}[1]{\ensuremath{\nabla^2 #1}}
